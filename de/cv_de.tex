%% LyX 2.1.4 created this file.  For more info, see http://www.lyx.org/.
%% Do not edit unless you really know what you are doing.
\documentclass[11pt,ngerman]{moderncv}
\renewcommand{\familydefault}{\sfdefault}
\usepackage[T1]{fontenc}
\usepackage{inputenc}
\usepackage{geometry}
\geometry{verbose,tmargin=3cm,bmargin=2cm,lmargin=2.5cm,rmargin=2.5cm}
\setcounter{secnumdepth}{2}
\setcounter{tocdepth}{2}
\setlength{\parskip}{\medskipamount}
\setlength{\parindent}{0pt}
\usepackage{graphicx}

\makeatletter

%%%%%%%%%%%%%%%%%%%%%%%%%%%%%% LyX specific LaTeX commands.
\providecommand{\LyX}{L\kern-.1667em\lower.25em\hbox{Y}\kern-.125emX\@}
%% Because html converters don't know tabularnewline
\providecommand{\tabularnewline}{\\}

%%%%%%%%%%%%%%%%%%%%%%%%%%%%%% User specified LaTeX commands.
\moderncvtheme[blue]{classic}
% Hier kann die Farbe geaendert werden
% possible themes are "classic" and "casual"
% optional argument are 'blue' (default), 'orange', 'red', 'green', 'grey' and 'roman' (for roman fonts, instead of sans serif fonts)

% the next part adds page-numbering
\usepackage{lastpage}
\rfoot{\addressfont\itshape\textcolor{gray}{\thepage\ von \pageref{LastPage}}}

% required
\firstname{Jan}
% required
\familyname{Claar}

% optional, remove the line if not wanted
\title{Lebenslauf} %Deutsch
%\title{Curriculum Vitae} %Englisch

% optional
% \address{street and number}{postcode city}
% '\\' adds a line break
\address{Lindauer Straße}{86399 Bobingen}

% optional
%\phone{+43(0)999 9999}
% optional
\mobile{+49(0)15788758676}
% optional
%\fax{+43(0)999 7777}
% optional
\email{mail@jan-claar.de}
%check all contact info twice!
%no email like swagmachine@hotmail.com
% optional
%\homepage{www.mypersonalpage.de}

% optional
% \photo[height][framebordersize]{name}
% 'height' is the height the picture is resized to
% 'name' is the name of the picture file
% 'framebordersize' is the size of the border arount the picture
\photo[80pt][0pt]{portrait.jpg}
%the picture need to be in the same directory as the document.

% optional
%\quote{"Wichtig ist, dass man nie aufh�rt zu fragen" - Albert Einstein}

\makeatother

\usepackage{babel}
\begin{document}


\maketitle


\section{Persönliche Daten}

\cvitem{Name}{Jan Ulli Claar}


\cvitem{Geburtsdatum}{{\small{}05.01.1999}}


\section{Schule und Studium}

\cventry{seit Okt. 2020}{Studium: Bachelor Informatik}{Universität Augsburg}{}{Deutschland}{}

\cventry{seit Okt. 2018}{Studium: Bachelor Physik mit Nebenfach Informatik}{Universität Augsburg}{}{Deutschland}{}
\cvitem{}{Aufnahme ins Max-Weber Programm des Freistaats Bayern}

\vspace{4mm}

\cventry{Sep. 2009 -- Mai 2017}{Abitur: Gymnasium bei St.Anna}{}{Augsburg}{Deutschland}{}

\section{Praktische Erfahrungen}

\cventry{seit Jan. 2020}{Software- \& Webdeveloper}{Liwetec GmbH}{Bobingen}{Deutschland}{Webanwendungsentwicklung v.a. mit PHP und JavaScript}

\cventry{Okt. 2019 -- März 2020}{Tutor der Informatik}{Lehrstuhl für Informatik Universität Augsburg}{}{}{Tutorentätigkeit für Informatik I}

\cventry{Mai 2018 -- Dez. 2019}{Servicekraft}{jap. Restaurant Kurofune}{Augsburg}{Deutschland}{Servicetätigkeit und Management anderer Teilzeitkräfte im japanischen Restaurant Kurofune.}

\vspace{4mm}


\cventry{Sep. 2017 -- Feb. 2018}{Ehrenamtliche Praktika}{Marburger Mission}{Karuizawa \& Kobe}{Japan}{Ehrenamtliche Tätigkeit in den Gemeinden, Reparatur- und Umbautätigkeiten der Einrichtungen}

\cvitem{}{Japanisch-Unterricht vor Ort}


\vspace{4mm}

%\section{Abschlussarbeiten und Veroffentlichungen}


\section{Sprachkenntnisse}




\cventry{Deutsch}{Muttersprache}{}{}{}{}{\vspace{4mm}
}


\cventry{Englisch}{C2-Niveau auf dem Europäischen Referenzrahmen}{7 Jahre Schulausbildung mit mehreren Schüleraustauschen; selbständige Weiterbildung durch Auslandsaufenthalte}{}{}{}{}\vspace{4mm}

\cventry{Japanisch}{A2/B1-Niveau}{2-jähriger Privatunterricht bei Frau Kazuko Fujisaki, 6-monatiger Aufenthalt in Japan mit weiterem Unterricht vor Ort, täglicher Gebrauch bei der Arbeit als Servicekraft}{}{}{}\vspace{4mm}

\cventry{Französisch}{Verständigung im Alltag}{3 Jahre Schulausbildung mit Schüleraustausch nach Lyon}{}{}{}\vspace{4mm}


%\section{Zusatzausbildungen}





\section{Sonstige Fertigkeiten}




\cventry{EDV}{Linux, Windows, Microsoft Office, Adobe Lightroom \& Photoshop}{Gute Kenntnisse in MS Office und Windows (ECDL, Okt. 2013), Weitgehende Erfahrung in Adobe Lightroom und Photoshop durch regelmäßigen Gebrauch}{}{}{}{}

\vspace{4mm}



\cventry{Programmieren}{PHP, Typescript, Python, Java, C}{Kenntnisse in Java, Python und C durch Studium, Anwendung von PHP und Typescript im Beruf}{}{}{}{}

\vspace{4mm}




\section{Hobbys - Interessen}




\cvitem{Kochen}{Schon seit der Kindheit koche ich oft und besonders gerne für andere. V.a. hat es mir die Japanische Küche angetan.}

\vspace{4mm}


\cvitem{Photographie}{Seit meinem Japanaufenthalt bin ich begeisterter Hobbyphotograph und interessiere mich besonders für Astrophotographie.}\vspace{4mm}

\cvitem{Jugendarbeit in der evang. Gemeinde}{Über lange Zeit habe ich mich in meiner Kirchengemeinde als Jugendmitarbeiter betätigt und dabei Jugend- und Kindergruppen mitgestaltet.}

\vspace{10mm}





\noindent \begin{flushright}
\begin{tabular}{cl}
Augsburg, der \today\hspace{4mm} & \includegraphics[height=1.2cm]{Signature.png}\tabularnewline
\cline{2-2} 
 & Jan Claar\tabularnewline
\end{tabular}
\par\end{flushright}
\end{document}
